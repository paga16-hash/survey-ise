The focus is on the security aspects of autonomous on-roar motor vehicles,
in particular the cybersecurity threats, standards, and industry practices used to secure these systems.
This review will not cover the ethical implications or the differences between human-driven and autonomous vehicles in terms of accidents and fatalities.

\subsection{Historical roots}\label{subsec:historical-roots}

The development of autonomous vehicles is a significant milestone in the evolution of intelligent transport systems.
It is important to start with some historical context to understand the current state of autonomous vehicles and then,
the security implications.

Vehicle automation has roots dating back to 1918 (Pendleton et al., 2017),
with General Motors showcasing the first concept of an automated vehicle in 1939 (Shladover, 2018).
Initial R\&D efforts were led by General Motors and the Radio Corporation of America Sarnoff Laboratory in the 1950s
(Shladover, 2018).
From 1964 to 2003, various government and academic initiatives in the US, Europe,
and Japan focused on automated buses, truck platoons, and advanced driving systems
(Shladover, 2018).
A significant boost came from DARPA’s Grand Challenges Program in the 2000s \cite{darpa_grand_challenges_book},
where AVs first navigated desert terrains in 2005 and urban roads by 2007 (Pendleton et al., 2017; Shladover, 2018).

Since then, researchers have rapidly progressed in academia and industry.
Volvo, Tesla, Audi, BMW,
Mercedes-Benz and Nissan are some of the major car manufacturers that have invested in AV technology \cite{faisal2019understanding}.

\begin{figure}[!htb]
    \centering
    \includegraphics[width=0.7\linewidth]{figures/history}
    \caption{Historical timeline of autonomous vehicles.}
    \footnotesize{From \cite{ahangar2021survey} }
    \label{fig:history}
\end{figure}

\subsection{Autonomous Vehicles}\label{subsec:autonomous-vehicles}

As key parts of cyber-physical systems (CPS), vehicles are expected to effectively address current automotive challenges.
The shift towards vehicle-to-everything (V2X) in future mobility increases the complexity of security issues, particularly in terms of adaptability, dynamism, and self-awareness \cite{connected_vehicles_security_2023,bouchouia2023survey} .
While there have been notable advancements in security for intelligent mobility, the transition from static security approaches remains necessary.
For instance, vehicles relying on open software protocols and connecting with in-vehicle electric infrastructures play a vital role in strengthening their security framework.
With the integration of advanced sensor platforms, these vehicles essentially evolve into highly sophisticated seto of technologies influencing each other.
This transformation enables capabilities such as in-vehicle computing, fleet management, synchronization, telemetry, and seamless information sharing in urban mobility settings. CITAZIONE
Additionally, these vehicles will manage network traffic transactions through embedded systems.
Given their capabilities, connected, intelligent, and autonomous vehicles not only generate vast amounts of data but also function as mobile virtual data sources and intelligent mobile entities.


\subsection{Overview of AVs Architecture}\label{subsec:overview-on-avs-architecture}

This section provides an overview of the architecture of autonomous vehicles (AVs) and the key parts that enable their operation.
Moreover, an alternative architecture proposed in \cite{2023survey} will be briefly discussed.
The architecture of AVs is designed to integrate various hardware and software components that work together to perceive the environment, make decisions, and control the vehicle.

\cite{architecture}

\begin{figure}[!htb]
    \centering
    \includegraphics[width=0.7\linewidth]{figures/state-architecture}
    \caption{Hierarchical architecture of autonomous vehicles control.}
    \footnotesize{From \cite{architecture} }
    \label{fig:architecture}
\end{figure}

The current architecture is composed of layers of hardware and software that interact to enable the vehicle to operate autonomously.
In its simplest form, the architecture consists of three main layers: communication, security, network and components layer.

The proposed architecture extends the state-of-the-art providing four new components: monitoring, analysis, decision-making, and visualization.
Any services, processes, and communication are monitored
by the agents and analyzed by the process controllers.
A set of decision controllers act
on the information from the process controllers.
The decisions are archived in the black
box, while the analysis, report, and visualization layers are capable of both in-vehicle and
external virtual security operation center (VSOC) HMIs.

\begin{enumerate}
    \item \textbf{Perception}: This stage involves sensing the AV's surroundings using sensors like RADAR, LIDAR, cameras, and real-time kinetic (RTK). The information is processed by recognition modules, such as adaptive detection and recognition frameworks (ADAF), control systems, lane departure warning systems (LDWS), traffic sign recognition (TSR), obstacle recognition, and vehicle positioning modules. The processed data is then sent to the decision and planning stage.
    \item \textbf{Decision and Planning}: Using the data from the perception stage, this stage plans and controls the AV's motion and behavior. It handles tasks like path planning, action prediction, and obstacle avoidance based on real-time maps, traffic data, user inputs, and past information. It may also include a data log module for error tracking and future reference.
    \item \textbf{Control}: The control module receives instructions from the decision and planning stage and manages the physical functions of the AV, such as steering, braking, and acceleration.
    \item \textbf{Chassis}: This final stage interfaces with the mechanical components like the motors for the accelerator, brake, steering wheel, and gear. These components are controlled by the signals from the control module to execute the AV’s movements.
\end{enumerate}

\subsubsection{Levels of Automation}\label{subsubsec:levels-of-automation}
The level of driving automation is determined by the specific roles assigned to both the driving automation system feature and the human user in performing the dynamic driving task (DDT) and DDT fallback. \cite{sae_j3016_2021}
The manufacturer of the automation system defines the requirements, operational design domain (ODD), and operating characteristics of the feature, including its level of automation.
Additionally, the manufacturer outlines how the feature should be properly used, ensuring that its capabilities and limitations are clearly understood and followed during operation.
The Society of Automotive Engineers (SAE) has defined six levels of driving automation, ranging from no automation (Level 0) to full automation (Level 5) \cite{sae_j3016_2021}.

\begin{enumerate}
    \item \textbf{Level 0 (No Automation):} The human driver is responsible for all aspects of the dynamic driving task.
    \item \textbf{Level 1 (Driver Assistance):} The vehicle assists the driver with specific tasks, such as steering or acceleration.
    \item \textbf{Level 2 (Partial Automation):} The vehicle can control both steering and acceleration/deceleration simultaneously under certain conditions, but the driver must remain engaged and monitor the environment.
    \item \textbf{Level 3 (Conditional Automation):} The vehicle can perform all aspects of the DDT under certain conditions like in traffic jams or highway driving, but the driver must be ready to take over when prompted.
    \item \textbf{Level 4 (High Automation):} The user (become passenger) does not need to supervise the Automated driving system (ADS) or be receptive to a request to intervene while the
    ADS is engaged, restricted to some conditions (e.g. Google's Self-Driving Car \cite{teoh2017rage}).
    \item \textbf{Level 5 (Full Automation):} The vehicle can perform all aspects of the DDT under all conditions without human intervention.
\end{enumerate}

\begin{table}[ht]
    \centering
    \begin{tabular}{|c|l|}
        \hline
        \textbf{Acronym} & \textbf{Definition} \\ \hline
        ADS & Automated Driving System \\ \hline
        DDT & Dynamic Driving Task \\ \hline
        ODD & Operational Design Domain \\ \hline
        OEDR & Object and Event Detection and Response \\ \hline
    \end{tabular}
    \caption{Definitions of Key Acronyms in Automated Driving}
    \label{tab:acronyms}
\end{table}

Levels 0–2 are generally classified as driver-assisted systems, while Levels 3 and 4 are considered semi-automated, and Level 5 represents full autonomy.

\subsubsection{Communication: VANETs}\label{subsubsec:communication}

VANETs are an emerging sub-class of mobile ad-hoc networks capable of spontaneous
creation of a network of mobile devices/vehicles \cite{sheikh2019comprehensive} . VANETs can be used for vehicleto-vehicle (V2V) and Vehicle-to-Infrastructure (V2I) communication. The main
purpose of such technology is to generate security on the roads; for example, during
hazardous conditions such as accidents and traffic jam the vehicles can communicate with
each other and the network to share vital information. The main components of
VANET technology are: \textit{On-board unit (OBU)}, \textit{Road-side unit (RSU)}, \textit{Trusted Authority}.
In this case can be noted how the Trusted Authority (TA) acts as a key component in the VANET architecture, as it is responsible for managing the security of the network and ensuring that all vehicles are authenticated and authorized to communicate with each other.

Communications are essential for AVs to interact with other vehicles, infrastructure, and the cloud.
When speaking about communications in the AVs context, it is also important to consider and categorize the communication type in:
\begin{enumerate}
    \item Long-range communication
    \item Medium-range communication
    \item Short-range communication
\end{enumerate}

The V2V communication provides several advantages such as BSD,
FCWS, automatic emergency braking (AEB) and LDWS \cite{arena2019overview} .
Vehicle-to-Vehicle (V2V) technology involves wireless communication between motor vehicles, aiming primarily to prevent accidents by exchanging data such as position and speed.
This communication happens within an ad-hoc mesh network, which can either be fully connected (where all nodes are directly linked) or partially connected (where only some nodes are fully connected, and others connect to those they interact with most).
In both cases, messages can be relayed directly (single hop) or through multiple paths (multi-hop), ensuring robust communication even if some nodes fail.
While wired mesh networks were once expensive and difficult to implement, wireless technologies, such as WPANs, have made them more feasible today \cite{arena2019overview} .

Depending upon the number
of hops used for inter-vehicle communication, they are classified as single-hop (SIVC) or
Multi-hop (MIVC) systems. The SIVC can be used for short-range applications such as
lane merging, ACC, etc., whereas MIVC can be used for long-range communication such
as traffic monitoring \cite{zheng2020cooperative} .


The USDOT (Department of Transportation of the United States of America) has documented the
implementation and development of the most advanced applications, including the requirements of
systems and in particular the need to define security network requirements for V2V and its supporting systems \cite{dot2021v2v} .



Vehicle to Infrastructure (V2I):
Vehicle to Infrastructure is another type of communication as key component of the intelligent transportation system (ITS) that enables vehicles to communicate with roadside infrastructure, such as traffic lights, signs, and road sensors.
Vehicle-to-Infrastructure (V2I) represents the next evolution in Intelligent Transportation Systems (ITS). V2I technologies gather traffic data generated by vehicles and wireless transmit information, such as safety, mobility, or environmental advisories, from the infrastructure to the vehicle, aiding the driver.
State and local agencies are expected to install V2I systems either alongside or integrated with existing ITS equipment.
As a result, many V2I deployments could be eligible for similar federal-aid programs as ITS, provided the deploying agency meets certain requirements.
\cite{dot2024v2i}

Vehicle to Everything (V2X):
Vehicle-to-Everything (V2X) is a general communication model that generalize Vehicle-to-Vehicle (V2V) and Vehicle-to-Infrastructure (V2I) systems, incorporating interactions between vehicles and other entities, such as pedestrians (V2P) \cite{vehicle-to-pedestrian}, roadside equipment (V2R) \cite{vehicle-to-roadside}, and devices (V2D).
V2X aims to enhance road safety, particularly for vulnerable road users (pedestrians, cyclists, motorcyclists), by supporting communication between vehicles and pedestrians, reducing accidents.
Technologies like Pedestrian Collision Warning (PCW) use wireless communication modules like Wi-Fi, Bluetooth, and Near Field Communication (NFC) to alert pedestrians of danger.

\begin{figure}[!htb]
    \centering
    \includegraphics[width=0.7\linewidth]{figures/communication}
    \caption{VANETs}
    \label{fig:communication}
\end{figure}


conclusions:

We developed an efficient ID-CPPA signature scheme for VANETs using bilinear maps to speed up message authentication at Roadside Units (RSUs).
The scheme utilizes general one-way hash functions, which are computationally less intensive than map-to-point hash functions.
It also employs batch signature verification, allowing RSUs to authenticate multiple messages simultaneously, even in high-traffic conditions.
The scheme is secure against various attacks and has a lower computational cost compared to existing methods.
Future work will focus on designing a version of the scheme without bilinear maps to further enhance efficiency in V2V communications.


\subsubsection{Perception}\label{subsubsec:perception}
It is fundamental in AVs to perceive the environment and understand the context in which the vehicle is operating for obvious reasons.
The perception system is responsible for collecting data from various sensors, such as cameras, LiDAR, radar, and ultrasonic sensors, to create a detailed representation of the vehicle's surroundings.
In this case, it is important to consider the possible tempering that can be done on this sensor to alterate the perception of the vehicle, leading to dangerous situations.

In the next sections there will be a deep dive into the main sensors used in AVs and their vulnerabilities.


\subsection{Cyber-insecurity consequences}\label{subsec:cyber-insecurity}

Before introducing new threats of new autonomous vehicle technologies, it is important to understand the current state of cybersecurity in AVs and some historical key attacks.
In the past, researchers have demonstrated various attacks on AVs, including remote hijacking, sensor spoofing, and data breaches.
As specified in \cite{cybersec} the attack surface of AVs is expanding due to the increasing complexity of these systems, which rely on a combination of hardware, software, and communication technologies.
This expansion is due to the aggressive attempts of automakers to
make vehicles fully autonomous in a short period of time and without considering the security implications.
This implied a lot of new technologies and new communication protocols keeping the focus always on the functionalities offered to the driver in terms of comfort.
The security of the vehicle has been considered as a secondary aspect, leading to a lot of vulnerabilities that can be exploited by attackers.

Attackers can exploit these vulnerabilities to compromise the safety and functionality of AVs, posing significant risks to passengers, pedestrians, and other road users.
Since now cars are connected to the internet, there is no need to be physically close to the car to exploit these vulnerabilities and this expands the attack surface drastically.
In the past decade (2010 onward), nearly 79.6\% of all automotive attacks have been
remote attacks.

Some famous attacks that bring the attention to the security of AVs are:
\begin{enumerate}
    \item The Jeep Cherokee hack in 2015, where researchers remotely hijacked a Jeep Cherokee through its infotainment system, demonstrating the potential risks of cyber-attacks on connected vehicles \cite{miller2015remote} .
    \item The Tesla Model S hack in 2016, where researchers exploited vulnerabilities in the vehicle's software to take control of the car's brakes, door locks, and other critical systems \cite{tesla_hack}.
    \item The Nissan Leaf hack in 2016, where researchers demonstrated how an attacker could remotely control the vehicle's heating and air conditioning systems, drain the battery, and access the driver's personal information.
    \item The VW group hack in 2016, where researchers discovered vulnerabilities in the keyless entry systems of several VW group vehicles, allowing attackers to unlock the doors and start the engine without the key fob \cite{garcia2016lock}.
\end{enumerate}

These are only some examples of the potential risks associated with AVs and the need for robust cybersecurity measures to protect against cyber-attacks.
One of the latest is the Tesla cybertruck vulnerabilities that have and continue to be exploited by attackers to gain access to the vehicle's systems and control its functions remotely.



























New sensing and communication capabilities bring the Autonomous vehicle to eexpose a big attack surface  \cite{unknown2020connected}
\cite{cybersec}

\subsection{Motivation}\label{subsec:motivation}

The motivation behind this survey is to provide a comprehensive overview of the current state of research on cybersecurity in autonomous vehicles.
The discussion begins with a brief history of autonomous vehicles (AVs) and an overview of some major attacks.
These incidents highlighted the importance of security in AV systems.
As a result, both manufacturers and researchers recognized the need to strengthen security measures.
This realization led to an emphasis on improving security by design.

There are some proposals like \cite{adu-kyere2023self-aware} that propose a self-aware security framework for AVs, and some standards like \cite{iso_21434_2021} that are trying to regulate the security of these systems, from the design to the deployment.
The problem is that the security of AVs is a complex issue to be solved by a single solution and without the collaboration of all the stakeholders.
The fact that more and more new technologies are being introduced in the automotive sector, like 5G, V2X communication, and AI, makes the security of these systems even more complex, and keeping the threshold of the security of these systems high is too complex without a by-design approach.
The goal is to provide a holistic vision of the main threats and the current solutions on which the researchers are focusing on the cybersecurity field analising main sensors and their vulnerabilites. without speak about the ethical implications of the technology or the comparison between human-driven and autonomous vehicles in terms of accidents and fatalities.


\subsection{Social implications}\label{subsec:social-implications}

This section aims to provide a little insight into the social implications of autonomous vehicles.
The introduction of autonomous vehicles (AVs) is expected to have a profound impact on society \cite{thomas2020perception}, transforming transportation systems \cite{intelligent_transportation_2023}
, urban planning \cite{impact_autonomous_vehicles_2018}, and the economy \cite{economic_aspects_2020}.

\begin{table}[ht]
    \centering
    \begin{tabular}{|l|l|}
        \hline
        \textbf{Advantages} & \textbf{Disadvantages} \\ \hline
        Casualty: AVs can significantly reduce the number of accidents. & Law: The definition of legal responsibilities can hinder the implementation of AVs. \\ \hline
        Fewer Expenses: Precise autonomous driving can reduce fuel consumption and increase the conservation of other parts. & Threat: AVs can be more vulnerable to network hacks because of the present computer-controlled functions. \\ \hline
        Productivity: The journey can be productive by performing other activities than driving. & Employment: There will be many job losses due to AVs in the transportation sector. \\ \hline
        Comfort: Interiors of AVs can be comfortable and spacious. & Price: The price of AVs is initially high, but after greater adoption, the price is going to decrease. \\ \hline
    \end{tabular}
    \caption{Advantages and Disadvantages of Autonomous Vehicles (AVs) from \cite{ahangar2021survey} }\label{tab:table}
\end{table}



\subsection{Structure of the Survey}\label{subsec:structure-of-the-survey}
