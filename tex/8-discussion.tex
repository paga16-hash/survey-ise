\subsection{Research Questions}\label{subsec:research-questions}
In this section a brief answer to the research questions is provided.
For each question one or more section are referred to where the answer is discussed in a more detailed way.

\subsubsection{Question 1}
\textit{What are the most significant cybersecurity threats facing autonomous vehicles, and how these threats evolve with advancements in vehicle-to-everything (V2X) communication and other emerging technologies?}

The most significant cybersecurity threats facing autonomous vehicles include sensor spoofing, remote hijacking, data breaches, and denial of service (DoS) attacks, a more detailed answer can be found in section \ref{sec:cybersecurity-threats-in-autonomous-vehicles}.
These threats evolve with advancements in vehicle-to-everything (V2X) communication and other emerging technologies, as discussed in section \ref{sec:future-directions}.
What can be understood is that the more the technolofy evolve the more the threats evolve and there is not wnough pressure on the security by design principles.
The definition of a rigorous security-by-design framework,  incapsulating current standards and best practices, is essential to address these threats in new technologies.
Another  important aspect is that the security of AVs is not only a technical issue, but also a socio-technical one, requiring collaboration among manufacturers, regulators, and other stakeholders to ensure comprehensive security measures.

\subsubsection{Question 2}
\textit{What existing standards and regulatory frameworks govern cybersecurity by design in autonomous vehicles, and how effectively do they address current and future security challenges?}
There are standards developed by organizations such as ISO/SAE and UNECE WP.29 that provide guidelines for cybersecurity by design in autonomous vehicles, as discussed in section \ref{sec:cybersecurity-by-design-standards}.
These standards aim to address current and future security challenges by establishing requirements for secure software development, risk assessment, and incident response.
However, there are gaps in the existing standards, such as the lack of specific guidance on emerging technologies like AI and blockchain, and the need for more comprehensive coverage of security across the entire AV lifecycle.
To effectively address current and future security challenges, it is essential to continuously update and expand existing standards to keep pace with technological advancements and evolving threats.

\subsubsection{Question 3}
\textit{What cybersecurity solutions and practices have been implemented by autonomous vehicle manufacturers, and how do they align with best practices in secure software development and system resilience?}
Autonomous vehicle manufacturers have implemented various cybersecurity solutions and practices, such as intrusion detection systems, secure over-the-air updates, and encryption, as discussed in section \ref{sec:implemented-cybersecurity-solutions}.
These practices align with best practices in secure software development and system resilience by integrating security measures into the software development lifecycle, monitoring for potential threats, and ensuring the integrity and confidentiality of data.
However, there are challenges in implementing these practices, such as the complexity of AV systems, the use of various technologies, and the need for continuous monitoring and updates to address new vulnerabilities.
The heterogeneity of AV systems and the interconnected nature of their components require a multi-faceted approach to security
that combines technical solutions with organizational processes and policies.
Collaboration among manufacturers to determine the best practices for secure software development and system resilience.
Motivated to the fact that heterogeneous systems will collaborate in the future, the need for a common ground is essential.

\subsubsection{Question 4}
\textit{What future trends and emerging technologies, such as artificial intelligence and blockchain, are being explored to enhance the cybersecurity of autonomous vehicles, and what challenges do they pose in real-world applications?}
Future trends and emerging technologies, such as artificial intelligence and blockchain, are being explored to enhance the cybersecurity of autonomous vehicles, as discussed in section \ref{sec:future-directions}.
Artificial intelligence is used for threat detection, adaptive security systems, and anomaly identification, while blockchain is employed maily for secure communication, data integrity, and identity management.
These technologies offer promising solutions to enhance the security of AVs, but they also pose challenges in real-world applications, such as scalability, latency, energy consumption, and privacy concerns.
To address these challenges, further research is needed to develop quantum-resistant cryptography, hybrid encryption techniques, and secure communication protocols that can withstand emerging threats.
