\subsection{Summary of Findings}\label{subsec:summary-of-findings}
Security of AV is an important issue and should be
considered systematically and holistically. In this paper, we
argued that the security-by-design principle for AV is poorly
understood and rarely practiced. We addressed the issue by
modeling an AV as a cyber-physical system and studied the
AV security objectives by viewing from the perspective of
a socio-technical framework. This was done methodically by
developing a security-by-design framework for AV from the
first principle.

We derived the security objectives and the necessary control
measures from the perspective of the safety requirements of
AV. We argued that one of the critical goals of cybersecurity
of AV is to ensure that safe operation is resilient in the face
of cyber-attacks. Subsequently, the technical challenges and
the proposed approaches for AV security were identified and
discussed.

Nevertheless, apart from safety assurance, for AV to be
adopted as a preferred means for transport, the legal and
liability issues behind AV remain a significant challenge. In
essence, technical designs and control measures should be
developed to enable law-enforcement agencies and judicial
officers to determine the liabilities and the parties at fault in the
unfortunate situation of car crashes which could lead to loss
of lives. The legal and liabilities issues are essential problems
that should be addressed as part of the future studies of AV
security.

According to Autonomous Vehicle: Security by Design




In this review, we have explored the critical importance of cybersecurity in the realm of
autonomous vehicles (AVs). The key points highlighted include the diverse and significant
threats faced by AV systems, such as remote hacking, sensor manipulation, data breaches,
and denial of service (DoS) attacks. We have also examined the existing countermeasures
deployed to combat these threats, including intrusion detection systems (IDSs), encryption,
regular updates, and robust authentication protocols. Furthermore, we have identified the
numerous challenges inherent in securing AVs, such as the complexity of AV systems, lack
of standardization, latency issues, and resource constraints.
The necessity of ongoing research and collaboration cannot be overstated. The rapidly
evolving landscape of cyber threats demands continuous innovation in cybersecurity technologies and practices. Collaborative efforts among AV manufacturers, cybersecurity
experts, regulatory bodies, and policymakers are essential to develop and implement
effective security measures. Sharing knowledge, standardizing protocols, and conducting joint research initiatives will help in addressing emerging threats and vulnerabilities
more efficiently.
Looking to the future, the vision for cybersecurity in autonomous vehicles is one
of resilience and adaptability. AI and machine learning will play pivotal roles in developing adaptive security systems capable of real-time threat detection and automated
response. Blockchain technology will enhance data integrity and secure communication
channels. Industry-wide collaboration and standardization will ensure consistent security
Electronics 2024, 13, 2654 16 of 21
practices across all AV platforms. Progressive legislative and policy developments will
provide the necessary regulatory framework to enforce robust cybersecurity measures and
protect consumers.
The journey towards secure autonomous vehicles is ongoing and multifaceted. By
embracing technological advancements, fostering collaboration, and enacting comprehensive policies, we can ensure that AVs are not only innovative but also safe and secure. This
holistic approach will pave the way for a future where autonomous vehicles can be trusted
to operate reliably, safeguarding both passengers and the broader public

accorrding to: Cybersecurity in Autonomous Vehicles—Are We Ready for
the Challenge?


Beyond the immediate safety risks, persistent sensor attacks can undermine public confidence in AV technology, slowing its adoption.
Ensuring the integrity and robustness of sensor data is crucial for maintaining trust in AVs.

according to: Cybersecurity in Autonomous Vehicles—Are We Ready for
the Challenge?
\cite{durlik2022cybersecurity}


\paragraph{Conclusion}:

We developed an efficient ID-CPPA signature scheme for VANETs using bilinear maps to speed up message authentication at Roadside Units (RSUs).
The scheme utilizes general one-way hash functions, which are computationally less intensive than map-to-point hash functions.
It also employs batch signature verification, allowing RSUs to authenticate multiple messages simultaneously, even in high-traffic conditions.
The scheme is secure against various attacks and has a lower computational cost compared to existing methods.
Future work will focus on designing a version of the scheme without bilinear maps to further enhance efficiency in V2V communications.





\subsection{Implications for Manufacturers, Policymakers, and Researchers}\label{subsec:implications-for-manufacturers-policymakers-and-researchers}
\subsection{Recommendations for Future Research}\label{subsec:recommendations-for-future-research}
