\subsection{Summary of Findings}\label{subsec:summary-of-findings}
Security of AV is an important issue and should be
considered systematically and holistically. In this paper, we
argued that the security-by-design principle for AV is poorly
understood and rarely practiced. We addressed the issue by
modeling an AV as a cyber-physical system and studied the
AV security objectives by viewing from the perspective of
a socio-technical framework. This was done methodically by
developing a security-by-design framework for AV from the
first principle.

We derived the security objectives and the necessary control
measures from the perspective of the safety requirements of
AV. We argued that one of the critical goals of cybersecurity
of AV is to ensure that safe operation is resilient in the face
of cyber-attacks. Subsequently, the technical challenges and
the proposed approaches for AV security were identified and
discussed.

Nevertheless, apart from safety assurance, for AV to be
adopted as a preferred means for transport, the legal and
liability issues behind AV remain a significant challenge. In
essence, technical designs and control measures should be
developed to enable law-enforcement agencies and judicial
officers to determine the liabilities and the parties at fault in the
unfortunate situation of car crashes which could lead to loss
of lives. The legal and liabilities issues are essential problems
that should be addressed as part of the future studies of AV
security.

According to Autonomous Vehicle: Security by Design


\subsection{Implications for Manufacturers, Policymakers, and Researchers}\label{subsec:implications-for-manufacturers-policymakers-and-researchers}
\subsection{Recommendations for Future Research}\label{subsec:recommendations-for-future-research}
