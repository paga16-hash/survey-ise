This review emphasizes the critical role of cybersecurity in autonomous vehicles (AVs), which are vulnerable to a range of cyber threats, including sensor tampering, hacking, and connectivity challenges.
While existing defenses like intrusion detection systems (IDS), regular updates, and current standards offer some protection, they are insufficient to fully safeguard AVs from evolving threats.

Key challenges include the lack of unified security standards, latency, scalability, and resource limitations, alongside a need for robust security-by-design principles and increased collaboration among manufacturers.
Persistent attacks can erode public confidence, slowing the adoption of AV technology.

To secure AVs, it's crucial to ensure data integrity, establish secure communication standards, and bolster sensor data resilience.
Knowledge sharing, standardizing protocols, and joint research are essential strategies to tackle emerging vulnerabilities effectively.

Looking ahead, AV cybersecurity will likely focus on adaptive, resilient systems that can withstand attacks.
Artificial intelligence, machine learning, and digital twins will be pivotal in developing real-time threat detection and automated response capabilities,
while blockchain technology will enhance data integrity and secure communications.
This comprehensive approach is vital to maintaining the safety, reliability, and public trust in autonomous vehicle systems.