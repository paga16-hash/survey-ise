\subsection{Conclusion}\label{subsec:conclusion}:

This review emphasizes the importance of cybersecurity for autonomous vehicles, which face a lot of cyber threats, including main and commons techniques, sensor tempering or connectivity challenges .
Existing defenses, like intrusion detection systems (IDS), regular software updates, and current standards are not enough to protect AVs from cyber threats.
Lack of uniform standards, latency, scalability and resource limitations, over a need for strict security by design and a missing manufacturer-collaborative approach has been identified as the main challenges.
Persistent possible attacks can erode public trust in autonomous vehicle technology, potentially delaying its broader adoption.
Maintaining the integrity of data, good and safe communications standards and resilience of sensor data are essential points to preserve confidence in AV systems and ensure their safe operation.
Knowledge sharing, standardizing security protocols, and joint research are key strategies to address new vulnerabilities effectively.

Future cybersecurity in AVs aims for systems that can adapt and withstand attacks.
Artificial intelligence, machine learning, digital twins will likely be central in developing real-time threat detection and automatic response systems, while blockchain technology will strengthen data integrity and communication security.

This comprehensive approach ensures that AVs remain safe and reliable as they advance, maintaining public trust and passenger safety.


This review underscores the critical role of cybersecurity in autonomous vehicles (AVs), which are vulnerable to a range of cyber threats, including sensor tampering, hacking, and connectivity challenges.
While existing defenses like intrusion detection systems (IDS), regular updates, and current standards offer some protection, they are insufficient to fully safeguard AVs from evolving threats.

Key challenges include the lack of unified security standards, latency, scalability, and resource limitations, alongside a need for robust security-by-design principles and increased collaboration among manufacturers.
Persistent attacks can erode public confidence, slowing the adoption of AV technology.

To secure AVs, it's crucial to ensure data integrity, establish secure communication standards, and bolster sensor data resilience.
Knowledge sharing, standardizing protocols, and joint research are essential strategies to tackle emerging vulnerabilities effectively.

Looking ahead, AV cybersecurity will likely focus on adaptive, resilient systems that can withstand attacks.
Artificial intelligence, machine learning, and digital twins will be pivotal in developing real-time threat detection and automated response capabilities,
while blockchain technology will enhance data integrity and secure communications.
This comprehensive approach is vital to maintaining the safety, reliability, and public trust in autonomous vehicle systems.