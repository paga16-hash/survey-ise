\subsection{Artificial Intelligence for Cybersecurity}\label{subsec:artificial-intelligence-for-cybersecurity-in-avs}
Artificial Intelligence (AI) and Machine Learning (ML)
are becoming increasingly important parts of cybersecurity in autonomous vehicles (AVs).
Given the vast amount of data generated by AVs,
many AI and ML applications can be found not only to perceive the environment
but also to enhance \textit{path following} algorithms, environmental sensing,
and improve driver convenience and safety\cite{giannaros2023autonomous}.

In the context of cybersecurity, AI is used to recognize threats,
deploy adaptive security systems, mitigate risks, identify anomalies,
and, in some instances,
automatically isolate potentially compromised components
or reroute communication paths to counter-threats\cite{durlik2022cybersecurity}.

\subsection{Blockchain for Secure Communication and Data Integrity}\label{subsec:blockchain-for-secure-communication-and-data-integrity}

\begin{enumerate}
    \item \textbf{Secure Communication}: Blockchain can be employed to secure vehicle-to-everything (V2X) communication,
    ensuring that the data exchanged between AVs, infrastructure, and other entities is tamper-proof and authenticated.
    This prevents unauthorized access and data manipulation.
    \item \textbf{Data Integrity}: The immutable nature of blockchain’s ledger ensures that all recorded data is verifiable
    and cannot be altered retroactively.
    This is particularly beneficial for maintaining the integrity of sensor data, driving logs, and software updates,
    thereby enhancing trust in the system.
    \item \textbf{Identity Management}: Blockchain can provide a decentralized framework for identity management,
    ensuring that only authenticated and authorized entities can access AV systems.
    This reduces the risk of identity spoofing and unauthorized access.
\end{enumerate}

Blockchain technology can also support \textbf{transparent and secure storage of data},
\textbf{securing communication channels}, \textbf{data integrity and privacy},
\textbf{forensics applications}, and \textbf{reputation and trust management}.
However, challenges remain in areas such as scalability, latency, energy consumption,
and privacy concerns
associated with both permission-less and permissioned blockchains\cite{bendiab2023autonomous, giannaros2023autonomous, khan2020cyber, admass2023cyber, ahmad2023machine}.

\subsection{Quantum-Resistant Cryptography}\label{subsec:quantum-resistant-cryptography-in-future-avs}
The rise of quantum computers presents a challenge to current cryptographic systems.
To defend against quantum-based attacks, quantum-resistant cryptography is essential.
Future research should prioritize
developing quantum-resistant algorithms to safeguard sensitive cryptographic data and ensure long-term security\cite{ahmad2023machine, admass2023cyber}.

\begin{enumerate}
    \item \textbf{Quantum-Resistant Cryptography}:
    Adopt post-quantum cryptographic algorithms (e.g., lattice-based, code-based) designed to withstand quantum attacks.
    \item \textbf{Quantum Key Distribution (QKD)}: Utilize quantum mechanics for secure key distribution,
    ensuring protection from quantum attacks in the V2X ecosystem.
    \begin{enumerate}
        \item Implement standardized components like quantum random number generators (QRNG).
        \item Use post-quantum encryption algorithms and reconfigurable hardware like FPGA.
    \end{enumerate}
    \item \textbf{Hybrid Cryptography}: Combine classical and quantum-resistant techniques to enhance security,
    using a hybrid approach for data encryption and key exchange.
    \item \textbf{Quantum-Safe TLS}:
    Develop or adopt quantum-resistant TLS protocols
    that use quantum-safe key exchanges and digital signatures for secure communication.
    \item \textbf{Quantum Blockchain Technology}: Use blockchain with quantum-resistant algorithms to secure transactions,
    ensuring data integrity and transparency.
    \item \textbf{Continuous Monitoring and Update}:
    Regularly assess and update security measures in CAV systems to stay ahead of quantum advancements.
\end{enumerate}

One interesting future direction is the integration of QC With Explainable
AI and Blockchain for Secure AVs\cite{bendiab2023autonomous}.


\subsection{Digital Twins and Cyber-Physical Security Testing}\label{subsec:digital-twins-and-cyber-physical-security-testing}
Digital twins can serve as a tool for enhancing cybersecurity measures in electric and autonomous vehicles.
By creating a virtual model that mirrors the vehicle's operations and interactions,
cybersecurity professionals can simulate attacks and assess vulnerabilities in a controlled environment.
This proactive approach allows for the identification and mitigation of security risks
before they impact the physical vehicle\cite{ali2023review}.

Another possible usage is in terms of testing.
Simulating environments, connections,
and attacks can help to understand the vulnerabilities of the system
and to test the effectiveness of the implemented security measures.
This can be done in a controlled environment, without the risk of damaging the physical vehicle.\cite{yigit2024cyber}

Proposal for digital twins not only for AVs but also for RSUs in VANETs environment,
focussing on a robust real-time vehicular network framework integrating cyber twins with advanced AI models,
addressing the complexities of vehicle-to-infrastructure communications\cite{yigit2024cyber}.