\subsection{Communication and Network}\label{subsec:v2x-communication-and-network}

DDoS -> cite this sontakke2022impact

Attackers can exploit these vulnerabilities to compromise the safety and functionality of AVs, posing significant risks to passengers, pedestrians, and other road users.
Since now cars are connected to the internet, there is no need to be physically close to the car to exploit these vulnerabilities and this expands the attack surface drastically.
In the past decade (2010 onward), nearly 79.6\% of all automotive attacks have been
remote attacks\cite{cybersec}.

\subsubsection{AMOEBA for location privacy}
\begin{enumerate}
    \item \textbf{Purpose}: Mitigates unauthorized tracking in VANETs using pseudonymization and random silent periods.
    \item \textbf{Key Features}:
    \begin{enumerate}
        \item Divides road networks into observed and unobserved (mixing) zones.
        \item Vehicles in unobserved zones change directions, speeds, and pseudonyms to prevent tracking.
        \item Uses navigation groups for anonymous access to Location-Based Services (LBS).
        \item Random silent periods between pseudonym updates in Vehicle-to-Vehicle (V2V) communications to unlink identities.
    \end{enumerate}
    \item \textbf{Limitations}: The leader of the navigation group may suffer privacy loss and computational overhead due to its role in the V2I applications.
\end{enumerate}

\subsubsection{Cryptographic MIX-Zone (CMIX)}
\begin{enumerate}
    \item \textbf{Purpose}: Protects privacy through cryptographic methods in mixing zones.
    \item \textbf{Key Features}:
    \begin{enumerate}
        \item Road-Side Units (RSUs) provide vehicles with public and private keys.
        \item Messages are encrypted inside mixing zones, and pseudonyms are exchanged for anonymous communication.
        \item Uses Pseudonym Certification Authority (PCA) for cross-certification and pseudonym validation.
        \item Hardware Security Modules (HSMs) and tamper-proof protections may be used to ensure security, though cost can be prohibitive.
        \item Rotating certificates to avoid correlation between different locations.
    \end{enumerate}
    \item \textbf{Challenges}: High cost of cryptographic hardware and managing cross-certification of pseudonyms.
\end{enumerate}

\subsubsection{K-Anonymity}
\begin{enumerate}
    \item \textbf{Purpose}: Ensures location privacy in LBS by making a vehicle’s position indistinguishable from others.
    \item \textbf{Key Features}:
    \begin{enumerate}
        \item \emph{K} refers to the minimum number of vehicles with similar location information.
        \item Utilizes a trusted central server that anonymizes data via spatial and temporal cloaking.
        \item The anonymity level (\emph{K}) and transmission of location data are customizable by users.
        \item A node only sends location information if there are \emph{K-1} other vehicles nearby.
    \end{enumerate}
    \item \textbf{Advantages}: Scalable and adaptable privacy protection based on the number of vehicles in proximity.
\end{enumerate}

\subsubsection{Dynamic Change MAC/PHY}
\begin{enumerate}
    \item \textbf{Purpose}: Protects privacy by dynamically changing network-layer identifiers in VANETs.
    \item \textbf{Key Features}:
    \begin{enumerate}
        \item Dynamically swaps MAC/PHY addresses on Wi-Fi networks (similar to DHCP dynamic IP addressing).
        \item Randomization of MAC/PHY addresses prevents long-term tracking by adversaries.
        \item Incorporates cryptographic key exchange to ensure security during the address swapping process.
    \end{enumerate}
    \item \textbf{Advantages}: Increases privacy by making it harder to track vehicles based on MAC/PHY addresses.
\end{enumerate}

\subsubsection{Density-Based Location Privacy (DLP)}
\begin{enumerate}
    \item \textbf{Purpose}: Enhances privacy based on the density of nearby vehicles.
    \item \textbf{Key Features}:
    \begin{enumerate}
        \item Vehicles change pseudonyms only when a certain number (\emph{k}) of neighboring nodes is reached.
        \item Operates with pre-equipped pseudonyms that are switched dynamically.
        \item The probability of tracking decreases as traffic density and speed variations increase.
    \end{enumerate}
    \item \textbf{Advantages}: Reduces location tracking probability in high-traffic areas, and improves performance compared to AMOEBA and Mix-Zone.
    \item \textbf{Challenges}: Optimal pseudonym change thresholds need to be determined dynamically.
\end{enumerate}

\subsubsection{Cybersecurity Issues and Countermeasures}
\begin{enumerate}
    \item \textbf{Fake Alerts}: False traffic alerts or CAMs can disrupt road operations. \textbf{Countermeasures}: Cryptographic authentication (e.g., ECDSA) to validate message legitimacy.
    \item \textbf{Data Theft}: Attackers can impersonate network nodes to capture sensitive packets. \textbf{Countermeasures}: Encryption and robust key management.
    \item \textbf{Unauthorized Profiling}: Misuse of personal data for profiling. \textbf{Countermeasures}: Pseudonymization to prevent linking identities to personal data.
    \item \textbf{Illusion Attacks}: False traffic warnings broadcast to create accidents or congestion. \textbf{Countermeasures}: Plausibility validation of messages.
    \item \textbf{Fake Identity \& Impersonation}: Adversaries impersonate legitimate vehicles for malicious activities. \textbf{Countermeasures}: ID-based cryptographic solutions to verify vehicle authenticity.
\end{enumerate}


\subsubsection{potential-threats-vanets}\label{subsubsec:potentials-threats-vanets}

\begin{table}[h]
    \centering
    \begin{tabular}{|l|l|l|}
        \hline
        \textbf{Attack} & \textbf{Compromised Services} & \textbf{Countermeasures} \\ \hline
        DOS & Availability, authentication & Use the bit commitment and signature-based authentication technique \\ \hline
        Jamming & Availability & Use frequency hopping technique, direct-sequence spread spectrum (DSSS) \\ \hline
        Malware & Availability & Reliable hardware and digital signature of software \\ \hline
        Broadcast tampering & Availability, integrity & Cryptographic primitives are enabled for prevention, but a non-repudiation mechanism may exist \\ \hline
        Blackhole, grayhole & Availability & Reliable hardware and digital signature of software \\ \hline
        Greedy behavior & Availability & Use intrusion detection systems (IDSs) \\ \hline
        Spamming & Availability, confidentiality & Reliable hardware and digital signature of software \\ \hline
        Eavesdropping & Confidentiality, integrity & Exploit physical layer security protocols \\ \hline
        Traffic analysis & Confidentiality & Use encryption techniques \\ \hline
        Man-in-the-middle & Authentication, confidentiality, integrity & Robust authentication technique such as digital certificates \\ \hline
    \end{tabular}
    \label{tab:Summary of Attacks and Countermeasures. }
\end{table}

table from \cite{sheikh2019comprehensive} .


This section highlights various attacks and threats that affect the security services in VANET systems, including availability, confidentiality, authentication, data integrity, and nonrepudiation.

\subsubsection{Attack on Availability}
\begin{enumerate}
    \item \textbf{Denial-of-Service (DoS) Attacks}: Block vehicle communication, disrupting operations; can occur as distributed denial of service (DDoS).
    \item \textbf{Jamming Attack}: Uses strong signals to disrupt communication channels, critical for safety applications.
    \item \textbf{Malware Attack}: Penetrates OBUs and RSUs, leading to system malfunctions.
    \item \textbf{Broadcast Tampering Attack}: Untrustworthy vehicles alter or replicate messages, hiding critical safety information.
    \item \textbf{Blackhole Attack}: Malicious nodes receive but do not forward packets, obstructing communication.
    \item \textbf{Grayhole Attack}: Selectively drops packets, similar to blackhole attacks.
    \item \textbf{Greedy Behavior Attack}: Misuses the message authentication code (MAC) protocol, causing traffic overload and collisions.
    \item \textbf{Spamming Attack}: Injects spam messages into the system, using bandwidth inefficiently.
\end{enumerate}

\subsubsection{Attack on Confidentiality}
\begin{enumerate}
    \item \textbf{Eavesdropping Attack}: Intercepts confidential information, such as user identity and location.
    \item \textbf{Traffic Analysis Attack}: Analyzes message transmission patterns to extract sensitive information.
    \item \textbf{Man-in-the-Middle Attack}: Intercepts and alters messages between communicating vehicles.
    \item \textbf{Social Attack}: Distracts drivers with unethical messages, affecting their performance.
\end{enumerate}

\subsubsection{Attack on Authentication}
\begin{enumerate}
    \item \textbf{Sybil Attack}: Uses multiple fake identities to disrupt normal operations and mislead other vehicles.
    \item \textbf{Tunneling Attack}: Links distant parts of the network, allowing unauthorized communication.
    \item \textbf{GPS Spoofing}: Creates false GPS information, misleading vehicles about their location.
    \item \textbf{Node Impersonation Attack}: Uses a valid user ID to impersonate another user.
    \item \textbf{Free-Riding Attack}: Exploits other users' authentication efforts without contributing.
    \item \textbf{Replay Attack}: Replays valid data to create unauthorized effects.
    \item \textbf{Key/Certificate Replication Attack}: Uses duplicates to confuse traffic authorities.
    \item \textbf{Message Tampering}: Alters exchanged messages in V2V or V2I communication.
    \item \textbf{Masquerading Attack}: Uses false identities to gain unauthorized access.
\end{enumerate}

\subsubsection{Attack on Data Integrity}
\begin{enumerate}
    \item \textbf{Masquerading Attack}: Uses valid credentials to broadcast false messages.
    \item \textbf{Replay Attack}: Repeats or delays transmissions, complicating emergency responses.
    \item \textbf{Message Tampering Attack}: Modifies messages to influence driving behavior.
    \item \textbf{Illusion Attack}: Generates misleading traffic warnings based on malicious data.
\end{enumerate}

\subsubsection{Attack on Non-repudiation}
\begin{enumerate}
    \item \textbf{Repudiation Attack}: Denies involvement in sending or receiving messages, complicating disputes.
\end{enumerate}


%//Cybersecurity and Privacy Protection in Vehicular Networks (VANETs)
%\cite{macena2023cybersecurity}
%// survey on VANETs sec serv.
%\cite{sheikh2019comprehensive}
%//Simulating attacks on Vanets
%\cite{simulation-attacks-vanets}


\subsection{Perception System}\label{subsec:perception-system}


\subsection{Software and Firmware}\label{subsec:software-and-firmware}
